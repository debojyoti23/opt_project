%%%%%%%%%%%%%%%%%%%%%%%%%%%%%%%%%%%%%%%%%
% Beamer Presentation
% LaTeX Template
% Version 1.0 (10/11/12)
%
% This template has been downloaded from:
% http://www.LaTeXTemplates.com
%
% License:
% CC BY-NC-SA 3.0 (http://creativecommons.org/licenses/by-nc-sa/3.0/)
%
%%%%%%%%%%%%%%%%%%%%%%%%%%%%%%%%%%%%%%%%%

%----------------------------------------------------------------------------------------
%	PACKAGES AND THEMES
%----------------------------------------------------------------------------------------

\documentclass{beamer}

\mode<presentation> {

% The Beamer class comes with a number of default slide themes
% which change the colors and layouts of slides. Below this is a list
% of all the themes, uncomment each in turn to see what they look like.

%\usetheme{default}
%\usetheme{AnnArbor}
%\usetheme{Antibes}
%\usetheme{Bergen}
%\usetheme{Berkeley}
%\usetheme{Berlin}
%\usetheme{Boadilla}
\usetheme{CambridgeUS}
%\usetheme{Copenhagen}
%\usetheme{Darmstadt}
%\usetheme{Dresden}
%\usetheme{Frankfurt}
%\usetheme{Goettingen}
%\usetheme{Hannover}
%\usetheme{Ilmenau}
%\usetheme{JuanLesPins}
%\usetheme{Luebeck}
%\usetheme{Madrid}
%\usetheme{Malmoe}
%\usetheme{Marburg}
%\usetheme{Montpellier}
%\usetheme{PaloAlto}
%\usetheme{Pittsburgh}
%\usetheme{Rochester}
%\usetheme{Singapore}
%\usetheme{Szeged}
%\usetheme{Warsaw}

% As well as themes, the Beamer class has a number of color themes
% for any slide theme. Uncomment each of these in turn to see how it
% changes the colors of your current slide theme.

%\usecolortheme{albatross}
%\usecolortheme{beaver}
%\usecolortheme{beetle}
%\usecolortheme{crane}
%\usecolortheme{dolphin}
%\usecolortheme{dove}
%\usecolortheme{fly}
%\usecolortheme{lily}
%\usecolortheme{orchid}
%\usecolortheme{rose}
%\usecolortheme{seagull}
%\usecolortheme{seahorse}
%\usecolortheme{whale}
%\usecolortheme{wolverine}

\setbeamertemplate{footline} % To remove the footer line in all slides uncomment this line
%\setbeamertemplate{footline}[page number] % To replace the footer line in all slides with a simple slide count uncomment this line

%\setbeamertemplate{navigation symbols}{} % To remove the navigation symbols from the bottom of all slides uncomment this line
}

\usepackage{graphicx} % Allows including images
\usepackage{booktabs} % Allows the use of \toprule, \midrule and \bottomrule in tables
\usepackage{amsfonts}
\usepackage{amsmath}
\DeclareMathOperator*{\argmax}{arg\,max}

%----------------------------------------------------------------------------------------
%	TITLE PAGE
%----------------------------------------------------------------------------------------

\title[Short title]{NON-DECOMPOSABLE PERFORMANCE MEASURES} % The short title appears at the bottom of every slide, the full title is only on the title page

\author{Debojyoti Dey (15511264)\\ Nimisha Agarwal (15511267)\\ (Group 15)} % Your name
%\institute[UCLA] % Your institution as it will appear on the bottom of every slide, may be shorthand to save space
%{
%University of California \\ % Your institution for the title page
%\medskip
%\textit{john@smith.com} % Your email address
%}
%\date{\today} % Date, can be changed to a custom date

\begin{document}
	
	\begin{frame}
		\titlepage % Print the title page as the first slide
	\end{frame}
	
	\begin{frame}
		\frametitle{Overview} % Table of contents slide, comment this block out to remove it
		\tableofcontents % Throughout your presentation, if you choose to use \section{} and \subsection{} commands, these will automatically be printed on this slide as an overview of your presentation
	\end{frame}

%----------------------------------------------------------------------------------------
%	PRESENTATION SLIDES
%----------------------------------------------------------------------------------------

%------------------------------------------------
\section{What we proposed?} % Sections can be created in order to organize your presentation into discrete blocks, all sections and subsections are automatically printed in the table of contents as an overview of the talk
%------------------------------------------------

\begin{frame}
	\frametitle{Problem Statement}
	Finding general optimization techniques which can work on all non-decomposable performance measures.
\end{frame}

%------------------------------------------------

%------------------------------------------------
\section{Some Terminologies} % Sections can be created in order to organize your presentation into discrete blocks, all sections and subsections are automatically printed in the table of contents as an overview of the talk
%------------------------------------------------

%------------------------------------------------

\begin{frame}
	\frametitle{Misclassification}
	\begin{itemize}
		\item Can be represented as -
		\begin{equation*}
		\sum \frac{1-y_iy_i^*}{2}
		\end{equation*}
		\item Decomposable.
	\end{itemize}	
\end{frame}

%------------------------------------------------

\begin{frame}
	\frametitle{Multivariate}
	\begin{equation*}
	\begin{split}
		& \overline{h}:\overline{\mathcal{X}} \rightarrow \overline{\mathcal{Y}}\\
		& \overline{h}_w(\overline{x})=\argmax_{\overline{y}^*\in\mathcal{Y}}(w^T\psi({\overline{x},\overline{y}^*}))\\
		& \psi(\overline{x},\overline{y}^*) = \sum_{i=1}^n y_i^*x_i
	\end{split}
	\end{equation*}
\end{frame}

%------------------------------------------------

\begin{frame}
	\frametitle{Non-decomposable Performance Measures}
	\begin{itemize}
		\item Performance Measures which do not decompose linearly.
		\item In classification problems, error rates are generally used.
		\item But not useful in case of class imbalance.
		\item So, measures like -
		\begin{itemize}
			\item F-measure is used for text retrieval.
			\item A continuous function of TPR and TNR is used in class imbalanced classification settings.
		\end{itemize}
	\end{itemize}
\end{frame}

%------------------------------------------------

\begin{frame}
	\frametitle{Examples}
	\begin{block}{G-mean:}
		\begin{equation*}
			\sqrt{pq}
		\end{equation*}
	\end{block} 
	\begin{block}{Min}
		\begin{equation*}
			min(P,N)
		\end{equation*}		
	\end{block}
	\begin{block}{H-mean:}
		\begin{equation*}
			\frac{2PN}{P+N}
		\end{equation*}		
	\end{block}
\end{frame}

%------------------------------------------------

\begin{frame}
	\frametitle{Problems in Non-decomposable Performance Measures}
	\begin{itemize}
		\item Non-convex.
		\item Require an approximate upper bound on the raw performance measure curve given by some (surrogate) convex function which can be optimized easily.
	\end{itemize}
\end{frame}

%------------------------------------------------

%------------------------------------------------
\section{Existing Solutions} % Sections can be created in order to organize your presentation into discrete blocks, all sections and subsections are automatically printed in the table of contents as an overview of the talk
%------------------------------------------------

\begin{frame}
	\frametitle{SVM Approach to Optimize Non-Linear Performance Measures}
	\begin{itemize} 
		\item \textbf{Training Sample}
		\begin{equation*}
		\mathcal{S} = ((x_1, y_1), \cdots \cdots, (x_n, y_n)) \in (\mathcal{X}\times \mathcal{Y})^n
		\end{equation*}
		drawn i.i.d. according to some unknown probability distribution.
		\item Find a rule $h \in \mathcal{H}$ from hypothesis space $\mathcal{H}$ such that it optimizes the expected prediction performance
		\begin{equation*}
		R^\Delta(h) = \int \Delta ((h(x_1), \cdots, h(x_n)), (y_1, \cdots, y_n))dPr(S)
		\end{equation*}
	\end{itemize}	
\end{frame}

%------------------------------------------------

\begin{frame}
	\frametitle{SVM Approach to Optimize Non-Linear Performance Measures}
	\begin{itemize}
		\item Loss function $\Delta$ is not decomposable.
		\item Optimize the empirical loss function $\Delta$ over the entire sample set.
		\begin{equation*}
		\hat{R}_S^\Delta(h) = \Delta ((h(x_1), \cdots, h(x_n)), (y_1, \cdots, y_n))
		\end{equation*}
		\item For such problems, structural SVM is used.
	\end{itemize}
\end{frame}

%------------------------------------------------

\begin{frame}
	\frametitle{SVM Approach to Optimize Non-Linear Performance Measures}
	General approach to this problem based on SVM
	\begin{equation*}
	\begin{split}
	& \underset{w,\zeta \geq 0}{min} \quad \frac{1}{2}\|w\|^2+C\xi\\
	& s.t \quad \forall\overline{y}^* \in \overline{\mathcal{Y}}\setminus\overline{y}:w^T[\psi(\overline{x},\overline{y})- \psi(\overline{x},\overline{y}^*)]\geq \Delta(\overline{y}^*,\overline{y})-\xi\\
	& \Rightarrow \Delta(\overline{y}^*,\overline{y}) + w^T(\Sigma x_i{y_i}^* - \Sigma x_i{y_i}) \leq \xi\\
	& \Rightarrow \Delta(\overline{y}^*,\overline{y}) + (\Sigma ({y_i}^* - {y_i})w^Tx_i) \leq \xi
	\end{split}
	\end{equation*}
	where $\Delta(\overline{y}^*,\overline{y})$ is loss function.	
\end{frame}

%------------------------------------------------

\begin{frame}
	\frametitle{SVM Approach to Optimize Non-Linear Performance Measures}
	\begin{itemize}
		\item Considering concave performance measures
		\begin{equation*}
		P_\psi = \psi(TPR,TNR)
		\end{equation*}
		\item For any concave function $\psi$ and $\alpha, \beta \in \mathbb{R}$
		\begin{equation*}
			\psi^*(\alpha, \beta) = \underset{u,v\in\mathcal{R}}{inf}\{\alpha u + \beta v - \psi(u,v)\}
		\end{equation*}
		\item By concavity,
		\begin{equation*}
		\psi(u,v) = \underset{\alpha,\beta\in\mathcal{R}}{inf}\{\alpha u + \beta v - \psi^*(\alpha, \beta)\}
		\end{equation*}
	\end{itemize}
\end{frame}

%------------------------------------------------

%------------------------------------------------
\section{Problems in Existing Solutions} % Sections can be created in order to organize your presentation into discrete blocks, all sections and subsections are automatically printed in the table of contents as an overview of the talk
%------------------------------------------------

\begin{frame}
	\frametitle{SPADE Algorithm}
	\begin{itemize}
		\item Requires link functions to be L-Lipschitz.
		\item Works with non-Lipschitz functions when some restrictions are imposed on them.
	\end{itemize}
\end{frame}

%------------------------------------------------

%------------------------------------------------
\section{What have we achieved?} % Sections can be created in order to organize your presentation into discrete blocks, all sections and subsections are automatically printed in the table of contents as an overview of the talk
%------------------------------------------------

\begin{frame}
	\frametitle{Our Solution}
	Trying to remove the restrictions to give a general solution.
\end{frame}

%------------------------------------------------

%------------------------------------------------
\section{How we achieved it?} % Sections can be created in order to organize your presentation into discrete blocks, all sections and subsections are automatically printed in the table of contents as an overview of the talk
%------------------------------------------------

\begin{frame}
	\frametitle{}
	
\end{frame}

%------------------------------------------------

%------------------------------------------------
\section{Results} % Sections can be created in order to organize your presentation into discrete blocks, all sections and subsections are automatically printed in the table of contents as an overview of the talk
%------------------------------------------------

\begin{frame}
	\frametitle{}
	
\end{frame}

%------------------------------------------------

%------------------------------------------------
\section{Conclusion} % Sections can be created in order to organize your presentation into discrete blocks, all sections and subsections are automatically printed in the table of contents as an overview of the talk
%------------------------------------------------

\begin{frame}
	\frametitle{}
	
\end{frame}

%------------------------------------------------

\begin{frame}
	\frametitle{}
	\begin{equation*}
	\begin{split}
	& \hat{R}_S^\Delta(h) = \Delta ((h(x_1), \cdots, h(x_n)), (y_1, \cdots, y_n))\\
	& h(x,w) = \argmax_{y\in\mathcal{Y}}F(x,y;w)\\
	& F(x,y;w)=\langle w,\psi(x,y)\rangle\\
	& h_w(\overline{x},w)=\argmax_{\overline{y}\in\mathcal{Y}}(w^T\psi({\overline{x},\overline{y}}))\\
	& \underset{w,\zeta \geq 0}{min}= \frac{1}{2}\|w\|^2+C\zeta\\
	& s.t \quad \forall\overline{y}' \in \overline{\mathcal{Y}}\setminus\overline{y}^*:w^T[\psi(\overline{x},\overline{y})- \psi(\overline{x},\overline{y}^*)]\geq \Delta(\overline{y}^*,\overline{y})-\zeta
	\end{split}
	\end{equation*}
\end{frame}

%------------------------------------------------

%------------------------------------------------
\section{References} % Sections can be created in order to organize your presentation into discrete blocks, all sections and subsections are automatically printed in the table of contents as an overview of the talk
%------------------------------------------------

\begin{frame}
	\frametitle{References}
	\footnotesize{
		\begin{thebibliography}{99} % Beamer does not support BibTeX so references must be inserted manually as below
			\bibitem[Joachims:2005:SVM:1102351.1102399]{p1} Joachims, Thorsten
			\newblock A Support Vector Method for Multivariate Performance Measures.
			\newblock \emph{Proceedings of the 22Nd International Conference on Machine Learning}
			
			
			\bibitem[NIPS2014_5504]{p2} Narasimhan, Harikrishna and Vaish, Rohit and Agarwal, Shivani
			\newblock On the Statistical Consistency of Plug-in Classifiers for Non-decomposable Performance Measures.
			\newblock \emph{Advances in Neural Information Processing Systems 27}
			
			\bibitem[conf/icml/NarasimhanK015]{p3} Narasimhan, Harikrishna and Kar, Purushottam and Jain, Prateek
			\newblock Optimizing Non-decomposable Performance Measures: A Tale of Two Classes.
			\newblock \emph{ICML}
			
			
			
		\end{thebibliography}
	}
\end{frame}

%------------------------------------------------

\begin{frame}
	\Huge{\centerline{Questions?}}
\end{frame}

%----------------------------------------------------------------------------------------

\end{document}